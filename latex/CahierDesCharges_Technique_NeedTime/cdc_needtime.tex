\documentclass[
	article,			% article académique
	11pt,				% taille de la police
	oneside,			% impression imprimante
	a4paper,			% taille du paper
	chapter=TITLE,
	french,			% langue
	sumario=tradicional
	]{base_nt}

\usepackage{comment}

\usepackage{titlesec}
\usepackage{titletoc}

\usepackage{pgfgantt}
\usepackage{xcolor}

\definecolor{myforestgreen}{RGB}{40, 167, 69}
\definecolor{myred}{RGB}{220, 53, 69}

% Redéfinit le style des \part pour supprimer les pages générées
\titleclass{\part}{top} % Utilise le style de section "top" pour les parties
% Modifie l'espacement avant le titre de la partie
\titlespacing*{\part}{0pt}{0pt}{0pt}

\titleformat{\part}[display]{\Huge}{}{-2cm}{}

% commande simple pour afficher les commandes Latex sous forme de texte
\newcommand{\comando}[1]{\textbf{$\backslash$#1}}

\titulo{Cahier des charges Technique du projet NeedTime()}

\tituloestrangeiro{Projet par des étudiants de l'EPITA}

\autor{
Keany Vy, Antoine, Yanis, Félix, Hamza \thanks{Ce cahier des charges a été élaboré dans le cadre d'un projet réalisé à l'EPITA - (École Pour l'Informatique et les Techniques Avancées) \url{epita.fr}} 
\\[0.5cm]
\href{mailto:keany-vy.khun@epita.fr}{keany-vy.khun@epita.fr}, \href{mailto:antoine.mulot@epita.fr}{antoine.mulot@epita.fr}
\\[0.5cm]
\href{mailto:felix.despretz@epita.fr}{felix.despretz@epita.fr}, \href{ mailto:yanis.jerjini@epita.fr}{yanis.jerjini@epita.fr}, \href{mailto:hamza.khamchane@epita.fr}{hamza.khamchane@epita.fr}
\\[0.3cm]
17 octobre, 2023 }


\evento{EPITA Projet S2 - Promo 2028 - Studio NeedTime()} % Bandeau au dessus de toutes les pages
\local{France}
\data{Version 1.0}

\begin{document}

\selectlanguage{french}

\frenchspacing 

% ----------------------------------------------------------
% ELEMENTOS PRÉ-TEXTUAIS
% ----------------------------------------------------------

%---
%
% Se desejar escrever o artigo em duas colunas, descomente a linha abaixo
% e a linha com o texto ``FIM DE ARTIGO EM DUAS COLUNAS''.
% \twocolumn[    		% INICIO DE ARTIGO EM DUAS COLUNAS
%
%---

\maketitle

% ----------------------------------------------------------
% ELEMENTOS TEXTUAIS
% ----------------------------------------------------------
\textual

% ----------------------------------------------------------
% Introduction
% ----------------------------------------------------------
%\begin{comment}

\let\cleardoublepage\clearpage

\part{Répartition des Tâches}

\section{Tableau de répartition des tâches}

\subsection{Tâches, rôles, suppléants}

\begin{quadro}
	\caption{Répartition des tâches}
	\centering
	\begin{tabular}{|c|c|c|}
		\hline
		Membre   & Rôle du membre & Suppléant  \\
		\hline
		Keany Vy & Multijoueur et réseau / Site web & Yanis \\
		Antoine & Génération de la map / Implémentation jouabilité & Félix \\
		Félix & Modélisation 3D et objets & Hamza \\
            Yanis & Menu (interface), sound design / Direction artistique  & Keany Vy \\
            Hamza & Intéraction (map / joueurs), IA  & Antoine \\
		\hline
		Login responsable &  Login suppléant  & \\
            \hline
            keany-vy.khun@epita.fr & yanis.jerjini@epita.fr & \\
            antoine.mulot@epita.fr & felix.despretz@epita.fr & \\
            felix.despretz@epita.fr & hamza.khamchane@epita.fr & \\
            yanis.jerjini@epita.fr & keany-vy.khun@epita.fr & \\
            hamza.khamchane@epita.fr & antoine.mulot@epita.fr & \\
            \hline
	\end{tabular}
	\legend{Ce tableau réparti les tâches des membres de l'équipe dans le projet}
\end{quadro}

\section{Avancement et Planification}

\subsection{Tableau de planification d'avancement des tâches}

    \begin{center}
    \begin{ganttchart}[
        y unit title=0.4cm,
        y unit chart=0.5cm,
        vgrid,
        hgrid,
        title label anchor/.style={below=-1.6ex},
        title left shift=.05,
        title right shift=-.05,
        title height=1,
        progress label text={},
        bar height=0.7,
        group right shift=0,
        group top shift=.6,
        group height=.3,
        bar/.append style={fill=myforestgreen},
        bar/.append style={draw=black!50},
        bar incomplete/.style={fill=myred}
    ]{1}{24}
    
    % Labels
    \gantttitle{Diagramme de Gantt}{24} \\
    \gantttitle{Soutenance 22-29 janvier}{8}
    \gantttitle{Soutenance 18-22 mars}{8}
    \gantttitle{Soutenance 17-21 juin}{8} \\
    
    % Tâches
    \ganttbar[progress=15]{Site web}{1}{11} 
    \ganttbar[progress=0]{}{22}{24}\\
    \ganttbar[progress=0]{Conception des personnages}{1}{13}
    \ganttbar[progress=0]{}{23}{24}\\
    \ganttbar[progress=3]{}{1}{5}
    \ganttbar[progress=0]{Modélisation 3D}{12}{20} \\
    \ganttbar[progress=10]{Développement Lore}{1}{17}\\
    \ganttbar[progress=0]{Intereactions joueurs/carte}{3}{17}
    \ganttbar[progress=0]{}{22}{24}\\
    \ganttbar[progress=0]{Implémentation jouabilité}{5}{19}\\
    \ganttbar[progress=0]{Intelligence artificielle}{6}{20}\\
    \ganttbar[progress=0]{Animations}{7}{13}\\
    \ganttbar[progress=0]{}{1}{2}
    \ganttbar[progress=0]{Multijoueur}{9}{24} \\
    \ganttbar[progress=0]{}{1}{3}
    \ganttbar[progress=0]{Interface graphique du jeu}{14}{24}\\
    \ganttbar[progress=5]{Génération de la carte}{1}{6}
    \ganttbar[progress=0]{}{15}{24}\\
    \ganttbar[progress=0]{}{4}{8}
    \ganttbar[progress=0]{Textures / Rendu}{16}{24}\\
    \ganttbar[progress=0]{Sons / Effets spéciaux}{20}{24}
    
    \end{ganttchart}
    \end{center}
    \end{document}