\documentclass[
	article,			% article académique
	11pt,				% taille de la police
	oneside,			% impression imprimante
	a4paper,			% taille du paper
	chapter=TITLE,
	french,			% langue
	sumario=tradicional
	]{base_nt}

\usepackage{comment}

% commande simple pour afficher les commandes Latex sous forme de texte
\newcommand{\comando}[1]{\textbf{$\backslash$#1}}

\titulo{Cahier des charges du projet NeedTime()}

\tituloestrangeiro{Projet par des étudiants de l'EPITA}

\autor{
Keany Vy, Antoine, Yanis, Félix, Hamza \thanks{Ce cahier des charges a été élaboré dans le cadre d'un projet réalisé à l'EPITA - (École Pour l'Informatique et les Techniques Avancées) \url{epita.fr}} 
\\[0.5cm]
\href{mailto:keany-vy.khun@epita.fr}{keany-vy.khun@epita.fr}, \href{mailto:antoine.mulot@epita.fr}{antoine.mulot@epita.fr}
\\[0.5cm]
\href{mailto:felix.despretz@epita.fr}{felix.despretz@epita.fr}, \href{ mailto:yanis.jerjini@epita.fr}{yanis.jerjini@epita.fr}, \href{mailto:hamza.khamchane@epita.fr}{hamza.khamchane@epita.fr}
\\[0.3cm]
17 octobre, 2023 }


\evento{EPITA Projet S2 - 2023 - Studio NeedTime()} % Bandeau au dessus de toutes les pages
\local{France}
\data{Version 1.0}

\begin{document}

\selectlanguage{french}

\frenchspacing 

% ----------------------------------------------------------
% ELEMENTOS PRÉ-TEXTUAIS
% ----------------------------------------------------------

%---
%
% Se desejar escrever o artigo em duas colunas, descomente a linha abaixo
% e a linha com o texto ``FIM DE ARTIGO EM DUAS COLUNAS''.
% \twocolumn[    		% INICIO DE ARTIGO EM DUAS COLUNAS
%
%---

\maketitle

\insereAbstract{Notre mission en tant que développeurs est de concevoir un jeu de déduction sociale exceptionnel, où chaque joueur devra faire face à des choix cruciaux pour survivre dans un environnement hostile. Avec notre jeu, nous cherchons à créer une expérience de jeu immersive, mettant l'accent sur la coopération, la trahison et l'intrigue, tout en repoussant les limites de la narration interactive.

L'idée fondamentale derrière notre jeu est de plonger les joueurs au cœur d'une situation de survie inattendue : un groupe de 4 à 15 personnes échoue sur une île isolée à la suite d'un accident aérien, d'un naufrage ou d'autres scénarios mystérieux. Leur mission : survivre et échapper. Mais ici réside l'élément captivant de notre jeu : la dynamique sociale complexe.

Les joueurs devront s'entraider pour survivre, formant des alliances tout en restant vigilants face aux trahisons potentielles. Notre jeu est conçu pour stimuler l'interaction entre les joueurs, incitant à la communication, à la planification stratégique, et à l'exploration collaborative de l'île.

Il offrira une immersion totale grâce à des graphismes de qualité, un design tropical et une mécanique de jeu fluide. Notre équipe de développement se concentrera sur la création d'une île complexe et vivante, riche en détails et en opportunités d'interaction.

\newpage


Le jeu proposera un système de jour et de nuit, amplifiant la tension et permettant aux joueurs de planifier et d'exécuter des actes de trahison pendant les heures d'obscurité. La nuit deviendra le moment idéal pour mettre en œuvre des stratégies sournoises, telles que le sabotage discret et le vol d'objets précieux.

Enfin, il sera une expérience de jeu novatrice, où les joueurs devront faire des choix difficiles pour échapper à l'île. Un seul joueur uniquement aura la possibilité de partir, tandis que les autres devront rester, cherchant à survivre ou à mériter une place dans le prochain sauvetage.
}

\vspace{10cm}
\begin{figure}[ht]
	\caption{Logo du studio NeedTime()}
	\centering
	\includegraphics[width=0.4\linewidth]{logpapp.png}
	\legend{Site web: \url{https://needtime.pages.dev/}}
	
\end{figure}

% ----------------------------------------------------------
% ELEMENTOS TEXTUAIS
% ----------------------------------------------------------
\textual

\newpage

\customtableofcontents

% ----------------------------------------------------------
% Introduction
% ----------------------------------------------------------
%\begin{comment}
\newpage

\part{Introduction}
\section{Présentation du projet}

\subsection{Objectifs du cahier des charges}

Le présent cahier des charges a pour objectif de fournir un cadre détaillé et structuré pour le développement du projet Betray. Il sert de document de référence essentiel pour l'équipe de développement, le studio et toutes les parties prenantes impliquées dans le projet. Les objectifs spécifiques du cahier des charges sont les suivants :

\begin{enumerate}
    \item Définir la vision du projet : Le cahier des charges vise à clarifier la vision du jeu Betray, en mettant en évidence les éléments clés qui feront de ce jeu une expérience unique. Il établit le contexte, l'intrigue et les mécanismes de jeu, tout en soulignant l'importance de la déduction sociale et de la trahison comme éléments centraux de la jouabilité.

    \item Structurer le développement : Le cahier des charges divise le projet en sections claires, précisant les différentes étapes, les fonctionnalités et les exigences à réaliser. Il fournit un plan de travail organisé qui servira de feuille de route pour l'équipe de développement tout au long du processus.

    \item Assurer la compréhension partagée : Ce document a pour objectif de garantir que tous les membres de l'équipe et toutes les parties prenantes comprennent les objectifs, les exigences et les spécifications du projet Betray. Il favorise une vision partagée, essentielle pour la coordination efficace et la réussite du projet.

    \item Faciliter la communication : Le cahier des charges crée un langage commun pour les discussions et les échanges au sein de l'équipe et avec les parties prenantes. Il contribue à éviter les malentendus, à résoudre les problèmes rapidement et à maintenir une communication fluide tout au long du projet.

    \item Établir une base d'évaluation : Il sert de référence pour évaluer la conformité du projet aux exigences initiales, permettant ainsi de mesurer les progrès et de garantir que le produit final sera en production avec la vision définie.
    
\end{enumerate}

Efin, ce cahier des charges constitue un outil essentiel pour la planification, l'exécution et l'évaluation du projet Betray. Il a pour but de garantir que l'ensemble de l'équipe et des parties prenantes soient alignés sur les objectifs du projet, tout en fournissant une référence pour guider le développement vers le succès.

\subsection{Présentation générale du jeu}

Betray est un projet ambitieux qui vise à révolutionner le monde des jeux de plateau en proposant une expérience de jeu de déduction sociale immersive et unique. Dans un paysage de jeu saturé, Betray se distingue en mariant des éléments de survie, d'intrigue et de coopération pour créer une expérience de jeu captivante et complexe.

Il s'agit d'un jeu multijoueur en ligne qui emmène les joueurs sur une île mystérieuse, suite à un accident aérien, un naufrage ou d'autres scénarios intrigants. Le jeu se situe dans la catégorie des jeux de déduction sociale, où la coopération, la trahison et la déduction sont au cœur de l'expérience.

L'objectif principal du jeu est de survivre sur l'île tout en établissant des alliances, en collectant des ressources et en planifiant l'évasion. Les joueurs doivent s'entraider pour rassembler les éléments nécessaires pour envoyer un signal de SOS et s'échapper de l'île. Cependant, la trahison est une menace constante, car un seul joueur peut s'échapper, créant ainsi un dilemme intense de coopération et de méfiance.

Betray se démarque par les caractéristiques suivantes :
\begin{itemize}
    \item Jeu de survie immersif : Les joueurs sont plongés dans un environnement réaliste et vivant, où la survie dépend de la coopération et de la collecte de ressources.
    \item Déduction sociale : Betray met l'accent sur la déduction sociale, où les joueurs doivent analyser le comportement des autres pour identifier les alliés et les traîtres potentiels.
    \item Jour et nuit : Un cycle jour-nuit dynamique affecte la jouabilité, permettant aux joueurs de comploter pendant la nuit.
    \item Trahison et coopération : Les joueurs peuvent former des alliances, mais la trahison est un choix possible, créant une dynamique sociale complexe.
\end{itemize}

Nous sommes actuellement en phase de développement, avec des équipes dédiées à la conception, à la programmation, à la création artistique et à la narration. Nos objectifs incluent la création d'une expérience de jeu immersive, la mise en place d'une communauté engagée de joueurs et la mise en production du jeu sur les plateformes de distributions.

Enfin Betray est bien plus qu'un jeu; c'est avant tout une expérience. Nous avons l'ambition de créer un monde où la déduction sociale et l'interaction humaine sont au centre de l'expérience de jeu. Avec un concept unique, une équipe dédiée et une vision passionnée.

\section{Origine et nature du projet}

\subsection{Genèse de l'idée}

L'idée de Betray est née d'une fascination pour les jeux de déduction sociale et d'une aspiration à créer une expérience qui non seulement divertirait les joueurs, mais les plongerait dans une dynamique sociale riche. Nous avons puisé notre inspiration dans des jeux emblématiques qui mêlent la confiance et la trahison, et nous avons décidé de créer notre propre vision de ce genre unique.

Au fil des discussions et des réflexions, l'idée de Betray a évolué. Nous voulions que les joueurs ressentent l'intensité des décisions prises dans un environnement réaliste, tout en ajoutant des éléments de survie et d'intrigue pour renforcer l'immersion. Nous avons établi que le développement de Betray serait une opportunité d'explorer notre passion pour la programmation et de mettre en pratique nos compétences nouvellement acquises.

\subsection{Type de jeu : jeu de déduction sociale}

Betray s'inscrit dans la catégorie des jeux de déduction sociale. Il tire son inspiration des jeux qui mettent en avant la coopération, la méfiance, et la capacité des joueurs à décoder les intentions des autres. Notre objectif est de créer une expérience immersive où la déduction sociale est au cœur de l'ergonomie du jeu.

En tant qu'étudiants, nous avons la chance d'explorer ce genre de jeu unique qui mêle l'aspect technique de la programmation à la complexité des interactions sociales. Betray se veut un projet pédagogique et créatif qui nous permettra de mettre en pratique nos compétences en programmation tout en développant une compréhension plus approfondie de la création de jeux vidéo.

L'origine de ce projet réside dans notre désir d'apprendre et de nous perfectionner dans le domaine du développement de jeux vidéo. Nous avons choisi de créer un jeu de déduction sociale en raison de son potentiel pour nous immerger dans un processus de création complexe et stimulant. Betray représente une opportunité d'appliquer nos connaissances académiques à un projet concret et de repousser les frontières de nos compétences techniques.

\part{Secteurs concernés}
\section{Objet de l'Étude}

\subsection{Buts et intérêts du projet}

Dans le cadre de notre projet, notre objectif est d'étudier les aspects techniques et ludiques du développement d'un jeu de déduction sociale en ligne. En se concentrant sur le développement en C\# avec l'utilisation de Godot Engine, cette étude vise à atteindre plusieurs objectifs clés :

\begin{enumerate}
    \item Maîtrise du langage C\# : L'objectif principal de cette étude est de renforcer notre maîtrise de ce langage de programmation. En explorant les aspects techniques de la programmation à objet tout au long du développement de Betray, nous visons à acquérir une expertise solide dans ce langage essentiel pour la création de jeux.

    \item Compréhension de Godot Engine : Il s'agit d'un moteur de jeu open source et polyvalent. Notre étude se concentre sur son utilisation pour le développement de Betray. Nous cherchons à acquérir une compréhension approfondie de cet outil, y compris ses fonctionnalités, son utilisation et ses avantages dans le développement de jeux.

    \item Développement de compétences interdisciplinaires : le projet nécessite une collaboration étroite entre les membres de l'équipe, chacun apportant des compétences diverses, de la programmation à la conception artistique en passant par la narration. Notre étude vise à renforcer notre capacité à travailler en équipe et à tirer parti des compétences complémentaires de chaque membre.

    \item Compréhension du genre du jeu de déduction sociale : le projet appartient à la catégorie des jeux de déduction sociale. Cette étude nous permettra d'explorer en profondeur les caractéristiques de ce genre, y compris la mécanique de la déduction sociale, la dynamique des joueurs et la manière dont les éléments ludiques contribuent à une expérience immersive.
\end{enumerate}

L'objet de cette étude est donc de nous immerger dans le processus de développement d'un jeu de déduction sociale tout en consolidant nos compétences techniques. Nous cherchons à combiner la maîtrise du langage de programmation, la compréhension de Godot Engine et la création d'une expérience ludique cohérente dans le but de produire un jeu de haute qualité et d'acquérir une expérience précieuse dans le domaine du développement de jeux vidéo.

\subsection{Avantages pour l'équipe et individuellement}

Notre projet présente une multitude d'avantages tant pour les membres de l'équipe que pour chaque individu impliqué. Ces avantages vont au-delà de la simple création d'un jeu et touchent des domaines essentiels du développement personnel et professionnel :

\begin{enumerate}
    \item Développement technique : Pour l'équipe, le projet est une opportunité d'approfondir ses compétences techniques en programmation, en particulier dans le langage C\#. Chaque membre pourra perfectionner ses connaissances en développement de jeux vidéo, en utilisant le moteur 3D Godot Engine.

    \item Collaboration et travail d'équipe : le projet met en lumière l'importance de la collaboration interdisciplinaire. Les membres de l'équipe apprendront à travailler ensemble, à gérer un projet complexe et à exploiter les compétences complémentaires de chacun. Cela renforcera leur capacité à travailler en équipe dans des contextes professionnels variés.

    \item Compréhension de la déduction sociale : Le jeu Betray repose sur la déduction sociale, ce qui offre aux membres de l'équipe une occasion unique de comprendre en profondeur les mécanismes de ce genre de jeu. Cette compréhension peut être appliquée dans des contextes variés, de la psychologie à la résolution de problèmes.

    \item Apprentissage de Godot Engine : le projet est une opportunité d'explorer et de maîtriser l'utilisation de Godot Engine. Cette compétence peut être transférée à d'autres projets et domaines d'application.

    \item Gestion de projet : La gestion de projet est une compétence essentielle pour l'équipe. Betray permettra aux membres de développer leur capacité à planifier, organiser et suivre les progrès, des compétences utiles dans n'importe quel domaine professionnel.

    \item Réalisation personnelle : Chaque membre de l'équipe pourra voir sa vision prendre forme, devenant une source de fierté et de satisfaction personnelle. Cela renforcera leur confiance en leurs compétences et leur capacité à relever des défis.

    \item Préparation pour l'industrie du jeu : Pour les membres de l'équipe intéressés par une carrière dans l'industrie du jeu, Betray est une étape cruciale. Ils auront l'occasion d'acquérir de l'expérience dans le développement de jeux, de créer un portfolio professionnel et de se démarquer dans un secteur compétitif.

    \item Expérience de travail pratique : Betray nous offre une véritable expérience de travail dans le développement de jeux. Cela pourra constituer un atout significatif dans notre futur parcours professionnel.
\end{enumerate}

\part{État de l'Art}

\section{Les prémices du jeu}

Plongez dans un monde mystérieux et captivant, où l'intrigue et la survie se conjuguent au cœur d'une île isolée. L'univers de Betray se dévoile à travers des scénarios variés, propulsant les joueurs sur cette île inhospitalière suite à un accident aérien, un naufrage ou d'autres circonstances tout aussi énigmatiques. L'île, protagoniste silencieuse de cette aventure, permet un terrain de jeu riche et diversifié.

\textbf{Le contexte unique} : L'île est une mosaïque d'environnements qui s'étendent des plages ensoleillées aux forêts luxuriantes, en passant par des montagnes majestueuses et des grottes mystérieuses. Chaque zone possède ses propres ressources, ses dangers et ses secrets à découvrir, invitant les joueurs à explorer et à s'adapter à un monde qui ne cesse de révéler ses mystères.

\textbf{Jour et nuit, Deux visages de l'Île} : Le cycle jour-nuit est au cœur de l'expérience Betray. Les journées sont baignées de lumière, favorisant la coopération, l'exploration et la collecte de ressources essentielles. Mais lorsque la nuit tombe, l'atmosphère se teinte de suspicion et de méfiance. Les feux de camp crépitants deviennent le théâtre de discussions secrètes, de manœuvres furtives et de choix cruciaux.

\textbf{La dualité coopération-trahison} : Au cœur du jeu réside l'équilibre subtil entre la coopération et la trahison. Les joueurs doivent s'allier, partager des ressources et collaborer pour survivre sur l'île. Pourtant, le défi ultime surgit lorsqu'il est temps d'envoyer un signal de SOS. Un seul joueur aura le privilège de quitter l'île, lançant ainsi un dilemme complexe : faire confiance à ses alliés ou les trahir pour sa propre évasion.

\textbf{Les trésors dissimulés} : La survie sur l'île dépend de la collecte d'éléments essentiels tels que le bois, la nourriture, l'eau et bien d'autres ressources. Ces objets, jalousement gardés, sont à la fois le moteur de la coopération et l'étincelle de la trahison. Les joueurs devront rivaliser pour les récupérer, créant une dynamique sociale complexe.

Ces prémices dessinent un tableau intrigant et immersif, où la lumière du jour révèle la beauté de l'île, tandis que la nuit cache des intentions obscures. L'île devient un lieu de survie, de défis et de choix moraux, où chaque action a des conséquences. Les prémices de Betray plongent les joueurs dans un monde mystérieux où les émotions et les interactions humaines sont à l'honneur.

\section{Jeux similaires existants}

\subsection{Among Us}

\begin{exemplo}
    \href{https://fr.wikipedia.org/wiki/Among_Us}{Among Us} est un jeu en ligne très populaire, mettant en scène une équipe d'astronautes devant accomplir des tâches dans un vaisseau spatial. Cependant, parmi eux se cachent des imposteurs dont la mission est de saboter l'équipage. Les joueurs doivent collaborer pour identifier les imposteurs tout en préservant la sécurité du vaisseau. Un jeu qui repose sur la déduction sociale et la méfiance.
\end{exemplo}

\subsubsection{Points forts}

\begin{itemize}
    \item Jeu de déduction sociale : Among Us excelle dans la création d'une expérience centrée sur la déduction sociale, où les joueurs doivent analyser le comportement de leurs coéquipiers pour identifier les imposteurs.
    \item Facilité d'accès : Le jeu est facile à prendre en main, ce qui le rend accessible à un large public, y compris aux joueurs novices.
    \item Multijoueur et interaction sociale : Among Us favorise l'interaction sociale en ligne, incitant les joueurs à communiquer, collaborer et parfois trahir, créant ainsi des situations tendues et amusantes.
    \item Personnalisation des personnages : Les joueurs peuvent personnaliser leurs personnages avec des costumes, des couleurs, et des chapeaux, ajoutant une touche ludique et de personnalisation.
    \item Mises à jour fréquentes : Les développeurs mettent régulièrement à jour le jeu en ajoutant de nouvelles fonctionnalités, cartes et correctifs, maintenant ainsi l'engagement de la communauté.
    \item Adaptabilité : Among Us peut être joué sur plusieurs plateformes, y compris PC, mobile et consoles, ce qui élargit son accessibilité.
    \item Grande communauté : Une communauté active de joueurs, d'influenceurs et de créateurs de contenu soutiennent le jeu, contribuant à sa popularité continue.
\end{itemize}

\subsubsection{Fonctionnalités spécifiques}

\begin{itemize}
    \item Déduction et mécanique d'imposteur : Parmi les fonctionnalités clés, la mécanique d'imposteur se distingue. Les imposteurs doivent feindre l'accomplissement de tâches et saboter discrètement le vaisseau tout en évitant la détection. Les joueurs doivent affiner leurs compétences de déduction pour repérer les comportements suspects.
    \item Discussion en temps réel : Une caractéristique majeure d'Among Us est la discussion en temps réel lors des réunions d'urgence. Les joueurs peuvent discuter, défendre ou accuser d'autres joueurs, ajoutant une couche d'interaction sociale et de débat.
    \item Tâches et objectifs : Les joueurs doivent accomplir des tâches variées à bord du vaisseau. Ces tâches apportent de la variété à l'ergonomie du jeu et servent de toile de fond à l'intrigue.
    \item Personnalisation : Among Us propose des options de personnalisation pour les personnages, avec une variété de couleurs, de costumes et de chapeaux, permettant aux joueurs de se démarquer et de s'exprimer.
    \item Mises à jour régulières : Les développeurs d'Among Us continuent à enrichir le jeu avec de nouvelles cartes, des fonctionnalités améliorées et des correctifs, assurant une expérience de jeu en constante évolution.
    \item Communauté active : La communauté d'Among Us est très active, contribuant à la création de mods, de vidéos et de contenus personnalisés. Cela élargit la portée du jeu et l'entretient.
    \item Jouabilité multiplateforme : Among Us offre une jouabilité multiplateforme, permettant aux joueurs sur PC, mobiles et consoles de jouer ensemble, ce qui renforce la diversité de la communauté.
\end{itemize}

\part{Le Studio NeedTime()}

\section{Nom et logo du studio}

\subsection{Origine du nom et du logo}

Le nom du studio NeedTime() est né d'une réflexion sur l'élément temporel central de l'ergonomie de notre jeu Betray. Il reflète la nécessité urgente du temps dans le jeu, car les joueurs doivent prendre des décisions cruciales pour leur survie dans un laps de temps limité. Le nom évoque également l'idée que chaque seconde compte, une notion cruciale dans les jeux de déduction sociale où les alliances peuvent être fragiles et les trahisons rapides.

Le logo du studio est une horloge qui pointe vers le symbolique nombre 42. Cette référence subtile à 42 rappelle le concept populaire issu de la science-fiction, du livre "\href{https://fr.wikipedia.org/wiki/La_grande_question_sur_la_vie,_l%27Univers_et_le_reste}{Le Guide du voyageur galactique}" de Douglas Adams, où le nombre 42 est la réponse à la question fondamentale de la vie, de l'univers et de tout le reste. Cela ajoute une touche d'humour et de mystère à notre identité.

L'utilisation des couleurs dans notre logo est soigneusement réfléchie pour refléter notre vision et notre message marketing :

\begin{itemize}
    \item Gris (Arrière-plan du logo) : Le gris symbolise la neutralité et la stabilité. Il représente le contexte stable de notre jeu; l'île et met en évidence l'élément constant du temps qui s'écoule. Il évoque également le calme avant la tempête, car l'île devient le théâtre de la méfiance et de la trahison une fois la nuit tombée.
    \item Rouge (Texte sur l'horloge) : Le rouge est une couleur puissante et énergique. Elle évoque l'urgence et l'importance du temps, soulignant l'aspect essentiel du temps dans le jeu. Le rouge évoque également l'émotion et l'intensité, qui sont des éléments clés de l'expérience de jeu.
    \item Blanc (Fond de l'horloge) : Le blanc est associé à la pureté, à la clarté et à la vérité. Il représente la phase diurne du jeu, où la coopération et la communication entre les joueurs sont essentielles pour avancer. Il évoque également l'idée de la lumière qui éclaire l'obscurité de la nuit.
    \item Noir (Chiffres de l'horloge) : Le noir symbolise le mystère et la profondeur. Il représente les heures sombres de la nuit où la trahison peut se cacher, ainsi que l'aspect complexe de la déduction sociale.
\end{itemize}

\subsection{Dépôt de marque à l'INPI}

Conscients de l'importance de protéger notre identité et notre propriété intellectuelle, nous prévoyons d'entreprendre des démarches pour le dépôt de marque à l'INPI. Ce dépôt de marque garantira que notre nom de studio, ainsi que notre logo distinctif, soient officiellement reconnus et protégés en tant que propriété intellectuelle.

Le dépôt de marque à l'INPI est une étape cruciale pour protéger nos droits exclusifs sur le nom de studio NeedTime() et sur notre logo. Cela nous permettra de l'utiliser sans crainte de contrefaçon et de garantir notre unicité sur le marché. De plus, cela renforce notre position en matière de propriété intellectuelle, ce qui peut être essentiel pour de futures négociations, partenariats et opportunités de développement.

Le processus de dépôt de marque à l'INPI sera délégué à un consultant externe, une équipe juridique pour garantir sa réussite. Cela démontre notre engagement à protéger notre identité visuelle et à préserver notre créativité pour le jeu Betray et pour d'autres projets futurs. Nous croyons que cela renforcera notre position sur le marché et contribuera à notre succès continu en tant que studio de développement de jeux.

\section{Spécificités de l'entreprise}

Le studio NeedTime() est actuellement constitué en tant que projet de développement de jeux créé par 5 étudiants de l'Epita. En tant que future entreprise, nous envisageons plusieurs options pour formaliser notre structure juridique. L'une des options les plus appropriées pour notre cas en tant q'étudiants serait la création d'une \href{https://fr.wikipedia.org/wiki/Association_loi_de_1901}{Association Loi 1901} en France.

\textbf{Association Loi 1901} : Le statut juridique est particulièrement adapté aux projets étudiants comme le nôtre. Il permet de créer une structure légale sans but lucratif, axée sur des activités à caractère éducatif, culturel ou caritatif. Le studio en tant qu'association pourrait ainsi bénéficier de certains avantages fiscaux et administratifs, notamment la possibilité de collecter des fonds pour le développement du jeu, d'accéder à des ressources et d'organiser des événements liés à notre projet.

En tant qu'étudiants, cette option offre également une flexibilité qui correspond à notre situation académique. Elle nous permet de gérer le projet de manière formelle, tout en garantissant une structure adaptée à notre stade de développement.

Nous continuerons à explorer les options juridiques pour la structure du projet, en s'adaptant à l'évolution de celui-ci et des besoins de l'équipe. Quelle que soit la forme juridique finale, notre objectif est de créer une entreprise qui favorise l'innovation, la créativité et la réussite dans le monde du développement de jeux vidéo.

\section{Secteur d'activité}

Le secteur d'activité du studio se situe dans l'industrie du développement de jeux vidéo, un marché en constante évolution. Notre entreprise se concentre sur la création de jeux de déduction sociale, une niche qui suscite un intérêt croissant parmi les joueurs.

Pour financer le développement de Betray et nos projets futurs,nous envisageons plusieurs sources de financement :

\begin{itemize}
    \item Investissements étudiants : Les membres de l'équipe investiront leurs ressources personnelles dans le projet, notamment du temps et des compétences techniques.
    \item Financement participatif : Pour impliquer la communauté de joueurs, nous envisageons des campagnes de financement participatif, telles que Kickstarter, pour collecter des fonds pour le développement du jeu. Cela peut également aider à valider l'intérêt du public pour notre projet.
\end{itemize}

Notre modèle économique repose sur la distribution numérique du jeu, associée à des ventes directes et à des microtransactions, ainsi qu'à des options de monétisation de contenu supplémentaire. Notre approche inclut :

\begin{itemize}
    \item Ventes de DLC : Les joueurs peuvent acheter le jeu en tant que jeu complet, offrant une expérience de jeu totale. Nous envisageons une structure de prix compétitive pour maximiser les ventes initiales.
    \item Microtransactions cosmétiques : Pour maintenir la rentabilité et garantir un support continu du jeu, nous proposerons des microtransactions pour des éléments cosmétiques, tels que des skins, des tenues ou des éléments de personnalisation.
    \item Extensions de contenu : Des extensions de contenu payantes pourront être introduites, ajoutant de nouvelles fonctionnalités, cartes, et mécanismes de jeu pour étendre l'expérience Betray.
    \item Contenu gratuit : Pour encourager la rétention de la communauté, nous prévoyons de fournir des mises à jour de contenu gratuites, telles que de nouvelles cartes et des correctifs.
\end{itemize}

\part{L'équipe}

\section{Présentation de l'Équipe}

\subsection{Biographie des membres de l'équipe}

\subsubsection{Keany Vy}

\subsubsubsection{Rôle dans le projet}

\subsubsection{Antoine}

\subsubsubsection{Rôle dans le projet}

\subsubsection{Félix}

\subsubsubsection{Rôle dans le projet}

\subsubsection{Yanis}

\subsubsubsection{Rôle dans le projet}

\subsubsection{Hamza}

\subsubsubsection{Rôle dans le projet}

\part{Répartition des Tâches}

\section{Tableau de répartition des tâches}

\subsection{Tâches}

\subsection{Personnes}

\subsection{Responsable}

\subsection{Suppléant}

\part{ Avancement et Planification}

\section{Tableau de planification d'avancement des tâches}

\subsection{Tâches}

\subsection{Soutenances}

\subsection{Pourcentage d'avancement}

\newpage

\bibliography{referencias}

\end{document}